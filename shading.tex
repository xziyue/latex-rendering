\documentclass{scrartcl}
\usepackage[T1]{fontenc}
\usepackage{amsmath,amssymb}
\usepackage{newtxtext,newtxmath}
\usepackage[scale=0.85]{tgcursor}
\usepackage{microtype}
\usepackage{babel}
\usepackage{hyperref}


\title{Phong Shading with \LaTeX}
\author{Alan Xiang}
\date{\today}


\begin{document}

\maketitle

\ExplSyntaxOn

% declare vectors
\newcommand{\DeclareVec}[2]
{
    \int_set:Nn \l_tmpa_int {\clist_count:n {#2}}
    \cs_if_exist:cF {#1}
    {
        \fparray_new:cn {#1} {\int_use:N \l_tmpa_int}
    }
    \int_set:Nn \l_tmpa_int {1}
    \clist_map_inline:nn {#2}
    {
        \fparray_gset:cnn {#1} {\int_use:N \l_tmpa_int} {##1}
        \int_incr:N \l_tmpa_int
    }
}

\newcommand{\DeclareConstFP}[2]
{
    \fp_if_exist:cF {#1}
    {
        \fp_new:c {#1}
    }
    \fp_gset:cn {#1} {#2}
}

\newcommand{\DeclareConstI}[2]
{
    \tl_if_exist:cF {#1}
    {
        \tl_new:c {#1}
    }
    \tl_gset:cn {#1} {#2}
}

% create a new binary output file
\tl_new:N \g_bin_filename_tl
\cs_new:Npn \bin_open:n #1
{
    \tl_gset:Nn \g_bin_filename_tl {#1}
    \iow_open:Nn \g_tmpa_iow {\jobname-bin.tmp}
}

\cs_new:Npn \bin_close:
{
    \iow_term:n {Converting~LaTeX~output~to~binary}
    \iow_close:N \g_tmpa_iow
    \exp_args:Nx \sys_shell_now:n {python3 ~ latex_output_to_binary.py ~ \jobname-bin.tmp ~ \g_bin_filename_tl}
}


% #1: value
\cs_new:Npn \bin_write_byte:n #1
{
    \iow_now:Nn \g_tmpa_iow {#1}
}


% #1: values
\cs_new:Npn \bin_write_byte_clist:n #1
{
    \clist_map_inline:nn {#1}
    {
        \bin_write_byte:n {##1}
    }
}


% writing (unsigned) binary value in small endian
% #1: value
% #2: number of bins
\int_new:N \l_bin_tmpa_int
\int_new:N \l_bin_tmpb_int
\cs_new:Npn \bin_write_bytes_se:nn #1#2
{
    \int_set:Nn \l_bin_tmpa_int {#1}
    \int_step_inline:nn {#2}
    {
        % get current byte
        \int_set:Nn \l_bin_tmpb_int {\int_mod:nn {\l_bin_tmpa_int} {256}}
        % proceed to next byte
        \int_set:Nn \l_bin_tmpa_int {\int_div_truncate:nn {\l_bin_tmpa_int} {256}}
        % write output
        \exp_args:Nx \bin_write_byte:n {\int_use:N \l_bin_tmpb_int}
    }
}

\cs_generate_variant:Nn \bin_write_bytes_se:nn {xn,xx}

% #1: width
% #2: height
\int_new:N \l_bin_tmpc_int
\cs_set:Npn \bin_write_bmp_header:nn #1#2
{
    % BMP file header

    \bin_write_byte:n {66} % B
    \bin_write_byte:n {77} % M
    
    % compute and write file size
    \int_set:Nn \l_bin_tmpc_int {54 + 3 * (#1) * (#2)}
    \bin_write_bytes_se:xn {\int_use:N \l_bin_tmpc_int} {4}

    \bin_write_byte_clist:n {0,0,0,0,54,0,0,0}

    % BMP info header
    \bin_write_byte_clist:n {40,0,0,0}
    \bin_write_bytes_se:xn {#1} {4} % width
    \bin_write_bytes_se:xn {#2} {4} % height
    \bin_write_byte_clist:n {1,0,24,0} % bytes per pixel
    \int_step_inline:nn {24}
    {
        \bin_write_byte:n {0}
    }
}


\tl_new:N \l_bmp_row_tl
\tl_new:N \l_bmp_col_tl
\int_new:N \l_bmp_tmpa_int
\fp_new:N \l_bmp_tmpa_fp
\tl_new:N \l_bmp_tmpa_tl
\newcommand{\RenderToBMP}[1]
{
    % we do not want to deal with padding (lazy), so just make sure width is mulitple of 4
    \int_compare:nF {\int_mod:nn {\ImageWidth} {4} = 0}
    {
        \GenericError{}{BMP~width~must~be~multiple~of~4}{}{}
    }

    \bin_open:n {#1}
    
    % write BMP header
    \bin_write_bmp_header:nn {\ImageWidth} {\ImageHeight}

    % write BMP data
    \int_step_variable:nNn {\ImageHeight} \l_bmp_row_tl 
    {
        \iow_term:x {Rendering~BMP~line~\l_bmp_row_tl}
        \int_step_variable:nNn {\ImageWidth} \l_bmp_col_tl 
        {
            % generate gradient image with gamma correction
            \fp_set:Nn \l_bmp_tmpa_fp { round( ( (\l_bmp_col_tl  / \ImageWidth)  ** 1.2) * 255.0 ) }
            \int_set:Nn \l_bmp_tmpa_int {\fp_use:N \l_bmp_tmpa_fp}
            \tl_set:Nx \l_bmp_tmpa_tl {\int_use:N \l_bmp_tmpa_int}
            \exp_args:Nx \bin_write_byte_clist:n {\l_bmp_tmpa_tl,\l_bmp_tmpa_tl,\l_bmp_tmpa_tl} % one pixel
        }
    }

    \bin_close:
}


\DeclareConstI{ImageHeight}{128}
\DeclareConstI{ImageWidth}{128}
\DeclareConstFP{kSpec}{0.8} % specular reflection constant
\DeclareConstFP{kDiff}{0.8} % diffuse reflection constant
\DeclareVec{AmbientColor}{0.05,0.05,0.05}
\DeclareVec{LightColor}{1.0,1.0,1.0}
\DeclareVec{ObjectColor}{1.0,0.0,0.0}

\RenderToBMP{render1.bmp}

\ExplSyntaxOff

\end{document}

